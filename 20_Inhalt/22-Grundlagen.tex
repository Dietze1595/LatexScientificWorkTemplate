\chapter{Grundlagen}\label{chap:Grundlagen}

In diesem Kapitel werden die Grundlagen vermittelt, die für das Verständnis dieser wissenschaftlichen Arbeit benötigt werden. Es werden lediglich Themenbereiche erläutert, welche nicht als Vorwissen zu erwarten sind. Da das Ergebnis der Arbeit für die Anwendung im Unternehmen \firma{AXOOM} spezialisiert sein wird, ist außerdem ein kurzer Unternehmensüberblick Teil der Grundlagen.

\section{Der Begriff \gqq{Industrie 4.0}}\label{sec:2_Industrie4}


Die erste industrielle Revolution erfolgte durch die Mechanisierung von Arbeitsschritten mittels Dampfkraft. Nach knapp 100 Jahren erfolgte die zweite industrielle Revolution durch die Verwendung von elektrischer Energie, die Massenanfertigungen an Fließbändern. Die dritte Revolution folgte mit der Automatisierungstechnologie durch den Einsatz von Elektronik und Computertechnik.

\begin{figure}[h]
	\centering
	\vspace{0.2cm}
	\includegraphics[width=1\textwidth]{50_Bilder/4-stufen.PNG}
	\caption[4 Stufen der industriellen Revolution]{4 Stufen der industriellen Revolution \cite{Kaufmann.2015}}\label{abb:stufen}
\end{figure}

Der Begriff Industrie 4.0 wurde erstmals von der Forschungsunion der deutschen Bundesregierung im Jahre 2013 geprägt. Um die Bedeutung dieses Wandels zu verdeutlichen, wurde das Ziel zum Ausdruck gebracht, eine 4. industrielle Revolution einzuleiten \cite{Kagermann.}. Angetrieben wird diese Entwicklung durch die Forschungsunion, ein Zusammenschluss aus hochrangiger Vertreter aus Wissenschaft und Wirtschaft \cite{.2018}.

In der vierte Revolution folgt die komplette Vernetzung der horizontale wie auch der vertikalen Integration. Der Wandel findet maßgeblich durch die Digitalisierung unter Anwendung der neuesten Kommunikations-, Informations- und Internettechnologien in industriellen Prozessen statt. Für \firma{AXOOM} ist ein Kernelement der \gqq{Industrie 4.0} die intelligente Kommunikation zwischen Mensch, Maschine und Produkt. Bestandteil davon ist die Verarbeitung von Prozessdaten und deren Verfügbarkeit über moderne Informationstechnologien. \firma{AXOOM} tritt an dieser Stelle mit einem umfangreichen Beratungs- und Produktangebot rund um die Bereitstellung sowie Verarbeitung von Prozessdaten auf den Markt.

\section{Unterschied API und SDK}\label{sec:2_API/SDK}

Ein \produkt{Application Programming Interface} kurz \gqq{\produkt{API}} genannt ist eine Schnittstelle, mit welcher unterschiedliche Softwareprogramme mit anderen Softwareprogrammen interagieren oder kommunizieren können. Ein \produkt{Software Development Kit} kurz \gqq{\produkt{SDK}} genannt ist eine Reihe von Tools, mit denen man eine Anwendung entwickeln kann. In den meisten Fällen enthalten SDK's mehrere API's.

Ein guter Vergleich im Alltag ist die Anwendung in einem Hause.
\par
\begingroup
\leftskip=1cm
\rightskip=1cm
\gqq{Das Haus selbst ist sozusagen die SDK. Sie ermöglicht die Erstellung der Anwendung. Für die Kommunikation innerhalb und außerhalb des Gebäudes wäre die API zuständig. In diesem Beispiel wären es die Telefonleitungen \cite{.2016}.}
\par
\endgroup

\section{Azure}\label{sec:2_Azure}	

\produkt{Microsoft Azure} nachfolgen \gqq{\produkt{Azure}} ist die Cloud-Plattform von \firma{Microsoft Corporation}, nachfolgend \gqq{\firma{Microsoft}}. Ende des Jahres 2008 wurde mit den Entwicklungsarbeiten für \produkt{Azure} bei \firma{Microsoft} begonnen. Seit dem 1. Februar 2010 steht die Plattform für die Kunden offiziell zur Verfügung. Seit dem Realese der Plattform bis heute werden immer wieder neue Dienste angeboten. Diese Dienste können in die bereits vorhandenen Anwendungen integriert werden \cite{Wikipedia.21.10.2018}.

\produkt{Azure} bietet neben einem fertigem API Management Tool auch eine SDK an, mit dem das Management von Schnittstellen realisiert werden kann. Ein großer Vorteil für eine\produkt{Azure}-Subskription ist die Plattform an sich. \produkt{Azure} bietet im Gegensatz zu anderen Cloud-Anbietern eine Cloud-Computing-Plattform an. Dies bedeutet dass der Speicherplatz, Rechenleistung oder die Anwendungssoftware als Dienstleister über das Internet bereitgestellt wird. [Ref: Bachelorthesis Jonas Baireuthery]

\section{C\#}\label{sec:2_C}



\subsection{.NET-Anwendung}\label{sec:2_NET}

\subsection{MVVM}\label{sec:2_MVVM}




\newpage

\section{Unternehmensstruktur}\label{sec:2_Unternehmensstruktur}


\textit{\textbf{Der folgende Informationsgehalt stammt aus firmeninternen Quellen.}}\\

\firma{TRUMPF GmbH \& Co. KG} besitzt zurzeit eine Tochtergesellschaft für das Zukunftsthema \gqq{Industrie 4.0}, die unter der Marke \firma{AXOOM} auftritt.

Die Firma \firma{AXOOM} wurde bereits im Jahr 2015 gegründet und ist für die Entwicklung und den Betrieb von Softwarelösungen für die digitalisierte Datenanalyse und das Monitoring von Maschinendaten aus Produktionsumgebungen im Sinne der \gqq{Industrie 4.0} verantwortlich.

Ebenfalls ist die \firma{AXOOM GmbH} zuständig für den Vertrieb und die Implementierung der \gqq{Industrie 4.0}-Lösungen bei Kunden. Nahezu jeder Kundenkontakt vom ersten Interesse über die Anpassung von Software an spezielle Kundenbedürfnisse bis hin zur fertigen Installation wird von \firma{AXOOM} übernommen.

Des weiteren sollte erwähnt werden, dass \firma{TRUMPF GmbH \& Co. KG} neben \firma{AXOOM} auch die Mehrheitsanteile an der Firma \firma{XETICS} mit dem Softwareprodukt \produkt{XETICS Lean} hält. \produkt{XETICS Lean} bietet ein sogenanntes \gqq{Manufacturing Execution System} (\acs{MES}), ein Managementsystem für Fertigungsprozesse, welches bei Bedarf ebenfalls von \firma{AXOOM} vertrieben und beim Kunden implementiert wird. 

Um Daten aus Maschinen und Anlagen für die datenverarbeitende Software zur Verfügung zu stellen, bietet die zugekaufte Firma \firma{C-Labs Corp.} aus den USA das \produkt{AXOOM}-Gate mit Werkzeugen für Schnittstellen und Datenbanken.

%In der letzten Zeit gab es einige Umbrüche in der Firmenstruktur. Dieser Umbrüche kamen aus den vorher entstandenen drei Tochtergesellschaften von Trumpf, welche sich alle in Richtung \gqq{Industrie 4.0} orientiert hatten \gqq{\firma{AXOOM Solutions}}, \gqq{\firma{C-Labs Corp.}} und \gqq{\firma{AXOOM}}. Diese drei Tochtergesellschaften laufen nun vollständig unter dem Namen \firma{AXOOM GmbH} zusammen.

\newpage










