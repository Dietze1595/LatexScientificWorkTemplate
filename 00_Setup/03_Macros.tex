%-------------------------------------
% Stichpunkte
%
% i: punktuell (\item ...)
% e: nummeriert (\item ...)
% d: fett (\item[...]) mit Beschreibung dahinter
%-------------------------------------
\newcommand\numberthis{\addtocounter{equation}{1}\tag{\theequation}}

% --- Punktuell
\newcommand{\bi}{\begin{itemize}}
\newcommand{\ei}{\end{itemize}}

% --- Nummeriert
\newcommand{\be}{\begin{enumerate}}
\newcommand{\ee}{\end{enumerate}}

% --- Fettgeschrieben mit Beschreibung
\newcommand{\bd}{\begin{description}}
\newcommand{\ed}{\end{description}}



%-------------------------------------
% Sonderzeichen
% 
% „...“:	\gqq{...}
%-------------------------------------

\newcommand{\gqq}[1]{{\glqq}#1{\grqq}}



%-------------------------------------
% Firmen- und Marken-Namen
% 
% Firma [kursiv]:		\firma{...}
% Produktname [kursiv]:	\produkt{...}
% Marke [Kapitälchen]:	\marke{...}
%-------------------------------------

\newcommand{\firma}[1]{\textit{#1}}
\newcommand{\produkt}[1]{\textit{#1}}
\newcommand{\marke}[1]{\textsc{#1}}



%-------------------------------------
% Fußnote auf andere Fußnote referenziert
%
% \footnote{text\label{xy}}
% \ftnref{xy}
%-------------------------------------

\newcommand\ftnref[1]{\textsuperscript{\ref{#1}}}



%-------------------------------------
% Quelle am Ende eines Kapitels
%
% Quelle: ... 	= 	\quelle{...}
% Quellen: ... 	= 	\quellen{...}
%-------------------------------------

\newcommand\quelle[1]{\begin{footnotesize}\textit{Quelle: #1}\end{footnotesize}}
\newcommand\quellen[1]{\begin{footnotesize}\textit{Quellen: #1}\end{footnotesize}}