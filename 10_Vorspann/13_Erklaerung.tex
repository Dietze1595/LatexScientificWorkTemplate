\chapter*{\centerline{Eigenständigkeitserklärung}}\label{chap:Erklaerung}

Ich versichere hiermit, dass ich die nachfolgende Arbeit mit dem Thema: \gqq{\produkt{Themenüberschrift}} selbstständig und ohne fremde Hilfe verfasst und keine anderen als die im Literaturverzeichnis angegebenen Hilfsmittel verwendet habe. Insbesondere versichere ich, dass ich alle wörtlichen und sinngemäßen Übernahmen aus anderen Werken als solche kenntlich gemacht und mit genauer Quellenangabe dargelegt habe. Die Arbeit hat mit gleichem Inhalt bzw. in wesentlichen Teilen noch keiner anderen Prüfungsbehörde vorgelegen.

\vspace*{2cm}

\hfill Musterstadt, den 1. Januar 1911, ~ \rule{0.3\textwidth}{0.4pt}

\hfill\parbox{0.20\textwidth}{\footnotesize{Max Mustermann}}

\newpage
\thispagestyle{empty}
\chapter*{\centerline{Geschlechtergerechte Sprache}}

Der besseren Lesbarkeit halber wird in dieser wissenschaftlichen Arbeit in der Regel die Sprachform des generischen Maskulinums angewendet. Es wird an dieser Stelle darauf hingewiesen, dass die ausschließliche Verwendung der männlichen Form geschlechtsunabhängig verstanden werden soll.
\newpage
\thispagestyle{empty}
~
\newpage